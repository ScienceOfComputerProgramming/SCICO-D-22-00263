%%%%%%%%%%%%%%%%%%%%%%%%%%%%%%%%%%%%%%%%%%%%%%%%%%%%%%%%%%%%
% UNCOMMENT THE LINE BELOW TO VIEW COMMENTS
\def \VersionWithComments {}
%%%%%%%%%%%%%%%%%%%%%%%%%%%%%%%%%%%%%%%%%%%%%%%%%%%%%%%%%%%%

%%
%% Copyright 2007, 2008, 2009 Elsevier Ltd
%%
%% This file is part of the 'Elsarticle Bundle'.
%% ---------------------------------------------
%%
%% It may be distributed under the conditions of the LaTeX Project Public
%% License, either version 1.2 of this license or (at your option) any
%% later version.  The latest version of this license is in
%%    http://www.latex-project.org/lppl.txt
%% and version 1.2 or later is part of all distributions of LaTeX
%% version 1999/12/01 or later.
%%
%% The list of all files belonging to the 'Elsarticle Bundle' is
%% given in the file `manifest.txt'.
%%
%% Template article for Elsevier's document class `elsarticle'
%% with numbered style bibliographic references
%% SP 2008/03/01

\documentclass[preprint,12pt, a4paper]{elsarticle}

%% Use the option review to obtain double line spacing
%% \documentclass[authoryear,preprint,review,12pt]{elsarticle}

%% For including figures, graphicx.sty has been loaded in
%% elsarticle.cls. If you prefer to use the old commands
%% please give \usepackage{epsfig}

\journal{Science of Computer Programming}

\usepackage[utf8]{inputenc}
\usepackage[english]{babel}


%%%%%%%%%%%%%%%%%%%%%%%%%%%%%%%%%%%%%%%%%%%%%%%%%%%%%%%%%%%%
% PACKAGES
%%%%%%%%%%%%%%%%%%%%%%%%%%%%%%%%%%%%%%%%%%%%%%%%%%%%%%%%%%%%
\usepackage[ruled,vlined,linesnumbered]{algorithm2e}
	\SetKwInOut{Input}{input}
	\SetKwInOut{Output}{output}

\usepackage{subcaption}

\usepackage{paralist} % inline lists

% Enumeration with (i)
\newenvironment{ienumerate}
	{\ifdefined\VersionLong\begin{enumerate}\else\begin{inparaenum}[\itshape i\upshape)]\fi}
	{\ifdefined\VersionLong\end{enumerate}\else\end{inparaenum}\fi}

% Enumeration with (1)
\newenvironment{oneenumerate}
	{\ifdefined\VersionLong\begin{enumerate}\else\begin{inparaenum}[1)]\fi}
	{\ifdefined\VersionLong\end{enumerate}\else\end{inparaenum}\fi}


	%%%%%%%%%%%%%%%%%%%%%%%%%%%%%%%%%%%%%%%%%%%%%%%%%%%%%%%%%%%%
% BEGIN BIBLATEX
%%%%%%%%%%%%%%%%%%%%%%%%%%%%%%%%%%%%%%%%%%%%%%%%%%%%%%%%%%%%
\ifdefined\VersionAuthor
	%%%%%%%%%%%%%%%%%%%%%%%%%%%%%%%%%%%%%%%%%%%%%%%%%%%%%%%%%%%%
	% bibLaTeX
	%%%%%%%%%%%%%%%%%%%%%%%%%%%%%%%%%%%%%%%%%%%%%%%%%%%%%%%%%%%%
	\usepackage[backend=biber,backref=true,style=alphabetic,url=false,doi=true,defernumbers=true,sorting=anyt,maxnames=99]{biblatex} % tlmgr install biblatex etoolbox logreq
	\addbibresource{PTASI.bib}

% 	% SOURCE: https://tex.stackexchange.com/questions/446139/multiple-bibliographies-using-multiple-files-with-biblatex
% 	\DeclareSourcemap{
% 		\maps[datatype=bibtex, overwrite]{
% 			\map{
% 			\perdatasource{Andre.bib}
% 			\step[fieldset=keywords, fieldvalue={, mypublications}, append]
% 			}
% 			\map{
% 			\perdatasource{AndreAddon.bib}
% 			\step[fieldset=keywords, fieldvalue={, mypublications}, append]
% 			}
% 		}
% 	}
	% Remove months and locations: SOURCE: https://tex.stackexchange.com/questions/89842/how-to-print-only-year-no-day-month-with-biblatex
% 	\AtEveryBibitem{\clearfield{month}}
	% \AtEveryBibitem{\clearfield{location}} % NOTE: does not work :((

	% Reformat DOI
	\renewbibmacro*{doi+eprint+url}{%
		\iftoggle{bbx:doi}
			{\color{black!40}\footnotesize\printfield{doi}}
			{}%
		\newunit\newblock
		\iftoggle{bbx:eprint}
			{\usebibmacro{eprint}}
			{}%
		\newunit\newblock
		\iftoggle{bbx:url}
	% 		{\color{black!40}\footnotesize\printfield{url}}
			{\usebibmacro{url+urldate}}
			{}%
	}

\fi
%%%%%%%%%%%%%%%%%%%%%%%%%%%%%%%%%%%%%%%%%%%%%%%%%%%%%%%%%%%%
% END BIBLATEX
%%%%%%%%%%%%%%%%%%%%%%%%%%%%%%%%%%%%%%%%%%%%%%%%%%%%%%%%%%%%


%%%%%%%%%%%%%%%%%%%%%%%%%%%%%%%%%%%%%%%%%%%%%%%%%%%%%%%%%%%%
% BEGIN Watermarking
%%%%%%%%%%%%%%%%%%%%%%%%%%%%%%%%%%%%%%%%%%%%%%%%%%%%%%%%%%%%
\ifdefined\VersionWithComments
	\usepackage{draftwatermark}
	\SetWatermarkText{draft}
	\SetWatermarkScale{15}
	\SetWatermarkColor[gray]{0.9}
\fi
% END Watermarking



%%%%%%%%%%%%%%%%%%%%%%%%%%%%%%%%%%%%%%%%%%%%%%%%%%%%%%%%%%%%
% HYPERLINKS
%%%%%%%%%%%%%%%%%%%%%%%%%%%%%%%%%%%%%%%%%%%%%%%%%%%%%%%%%%%%
\usepackage[svgnames,table]{xcolor}
\definecolor{darkblue}{rgb}{0, 0, 0.7}

\usepackage[
		pdfauthor={Author's name},%
		pdftitle={Document Title},
		breaklinks  = true,
		colorlinks  = true,
	\ifdefined \VersionWithComments
% 		pagebackref = true, % WARNING! incompatible with LNCS' hack by Nicolas Markey
	\fi
		citecolor   = blue!50!blue,
		linkcolor   = darkblue,
		urlcolor    = blue!50!green,
	]{hyperref}

%%%%%%%%%%%%%%%%%%%%%%%%%%%%%%%%%%%%%%%%%%%%%%%%%%%%%%%%%%%%
% CLEVER REFERENCES
%%%%%%%%%%%%%%%%%%%%%%%%%%%%%%%%%%%%%%%%%%%%%%%%%%%%%%%%%%%%
\usepackage[capitalise,english,nameinlink]{cleveref} % load after algorithm2e, amsthm and hyperref
\crefname{line}{\text{line}}{\text{lines}} % to remove the capital


%% The amssymb package provides various useful mathematical symbols
\usepackage{amssymb}
% \usepackage{hyperref}
%% The amsthm package provides extended theorem environments
%% \usepackage{amsthm}


%%%%%%%%%%%%%%%%%%%%%%%%%%%%%%%%%%%%%%%%%%%%%%%%%%%%%%%%%%%%
% COMMENTS MACROS
%%%%%%%%%%%%%%%%%%%%%%%%%%%%%%%%%%%%%%%%%%%%%%%%%%%%%%%%%%%%

\ifdefined\VersionWithComments
	\usepackage[colorinlistoftodos,textsize=footnotesize]{todonotes}
\else
	\usepackage[disable]{todonotes}
\fi
\newcommand{\gennote}[3]{\todo[size=\scriptsize,linecolor=#2,backgroundcolor=#2!25,bordercolor=#2]{#3: #1}\xspace}

% NOTE: version inline
% \newcommand{\gennote}[3]{\mbox{}\\\fcolorbox{#2!50!black}{#2!5}{%
% 	\begin{minipage}{.96\columnwidth}%
% 		{\small\color{#2}{\textbf{#3}: #1}\xspace}%
% 	\end{minipage}%
% }\\}


\newcommand{\ea}[1]{\gennote{#1}{blue}{ÉA}}
\newcommand{\bg}[1]{\gennote{#1}{orange}{BG}}
\newcommand{\reviewer}[2]{{\gennote{``#2''}{purple}{Reviewer #1}}}
% HACK but doesn't work :( TODO
% \def \oldtodo \todo
% \renewcommand{\todo}[1]{\oldtodo{#1}\xspace}

% Sometimes, we just need the old-style TODO!
\ifdefined \VersionWithComments
	\newcommand{\todoinline}[1]{\mbox{}{\color{red}{\textbf{TODO}\ifx#1\\\else:\ \fi #1}}} % here, ``\\'' stands for ``empty''
	\newcommand{\instructions}[1]{\mbox{}{\color{blue}{\footnotesize\textbf{Instructions}\ifx#1\\\else:\ \fi #1}}} % here, ``\\'' stands for ``empty''
\else
	\newcommand{\todoinline}[1]{}
	\newcommand{\instructions}[1]{}
\fi

\usepackage{verbatim} % for 'comment' environment


%%%%%%%%%%%%%%%%%%%%%%%%%%%%%%%%%%%%%%%%%%%%%%%%%%%%%%%%%%%%
% LINE NUMBERS
%%%%%%%%%%%%%%%%%%%%%%%%%%%%%%%%%%%%%%%%%%%%%%%%%%%%%%%%%%%%
\usepackage[pagewise]{lineno} % switch, modulo
\renewcommand\linenumberfont{\normalfont\tiny\sffamily\color{gray}}


%%%%%%%%%%%%%%%%%%%%%%%%%%%%%%%%%%%%%%%%%%%%%%%%%%%%%%%%%%%%
% MATH CONSTANTS
%%%%%%%%%%%%%%%%%%%%%%%%%%%%%%%%%%%%%%%%%%%%%%%%%%%%%%%%%%%%


% SETS
\newcommand{\setN}{\ensuremath{\mathbb N}}
\newcommand{\setQ}{\ensuremath{{\mathbb Q}}}
\newcommand{\setQplus}{\ensuremath{\setQ_{+}}} % \geq 0
\newcommand{\setR}{\ensuremath{\mathbb R}}
\newcommand{\setRgeqzero}{\ensuremath{\setR_{\geq 0}}}
\newcommand{\setRgzero}{\ensuremath{\setR_{>0}}}
\newcommand{\setRplus}{\ensuremath{\setR_{+}}} % \geq 0
\newcommand{\setZ}{\ensuremath{\mathbb Z}}




%%%%%%%%%%%%%%%%%%%%%%%%%%%%%%%%%%%%%%%%%%%%%%%%%%%%%%%%%%%%
% STRING CONSTANTS
%%%%%%%%%%%%%%%%%%%%%%%%%%%%%%%%%%%%%%%%%%%%%%%%%%%%%%%%%%%%
\newcommand{\imitator}{\textsf{IMITATOR}}


%%%%%%%%%%%%%%%%%%%%%%%%%%%%%%%%%%%%%%%%%%%%%%%%%%%%%%%%%%%%
% I.E. / E.G. / W.R.T.
%%%%%%%%%%%%%%%%%%%%%%%%%%%%%%%%%%%%%%%%%%%%%%%%%%%%%%%%%%%%

% Helps to spot the places where macros are NOT used
\ifdefined \VersionWithComments
 	\definecolor{colorok}{RGB}{80,80,150}
\else
	\definecolor{colorok}{RGB}{0,0,0}
\fi

\newcommand{\eg}{\textcolor{colorok}{e.g.,}\xspace}
\newcommand{\etal}{\textcolor{colorok}{\emph{et al.}}\xspace}
\newcommand{\ie}{\textcolor{colorok}{i.e.,}\xspace}
\newcommand{\st}{\textcolor{colorok}{s.t.}\xspace}
\newcommand{\viz}{\textcolor{colorok}{viz.,}\xspace}
\newcommand{\wrt}{\textcolor{colorok}{w.r.t.}\xspace}

% \renewcommand{\orcidID}[1]{\href{https://orcid.org/#1}{\includegraphics[height=1em]{ORCIDiD_icon128x128.png}}}

% NOTE: source = Springer (TACAS 2021)
\makeatletter
\def\orcidID#1{\smash{\href{https://orcid.org/#1}{\protect\raisebox{-1.25pt}{\protect\includegraphics{ORCID_Color.eps}}}}}
\makeatother


% Science of Computer Programming
% Software Track Template for Short Paper
% Before you complete this template, a few important points to note:
% *	This template is for the short paper associated with an original software.
% If you are submitting an update to a software that has already been published,
% please use the software update template.
% *	The nature of a short paper associated with a software submission is very
% different to a traditional research article. To help you write yours,
% we have created this template. We will consider only software accompanied by
% short papers written using this template. However, it is acceptable
% to rename the section titles to something that is more specific to
% your software.
% •	In particular, remember that the goal of the paper is that it supports the submission of your software. It’s your software that will be evaluated, and the accompanying paper is only a part of the process.
% •	We strongly advise to make this short paper not longer than 6 pages.
% *	It is mandatory to publicly share the code and software referred to in your
% software article. You'll find information on our software sharing criteria in
% the Guide for Authors:
% https://www.elsevier.com/journals/science-of-computer-programming/0167-6423/guide-for-authors
% *	It's important to consult the Guide for Authors when preparing your
% submission; it highlights mandatory requirements and is packed with useful advice.
%
% Still got questions? Email our editorial team at scico.editors@gmail.com.
%
%Now you are ready to fill in the template below. As you complete each section,
% please carefully read the associated instructions. All sections are mandatory,
% unless marked optional.

\begin{document}

\begin{frontmatter}

%% Title, authors and addresses

%% use the tnoteref command within \title for footnotes;
%% use the tnotetext command for theassociated footnote;
%% use the fnref command within \author or \address for footnotes;
%% use the fntext command for theassociated footnote;
%% use the corref command within \author for corresponding author footnotes;
%% use the cortext command for theassociated footnote;
%% use the ead command for the email address,
%% and the form \ead[url] for the home page:
%% \title{Title\tnoteref{label1}}
%% \tnotetext[label1]{}
%% \author{Name\corref{cor1}\fnref{label2}}
%% \ead{email address}
%% \ead[url]{home page}
%% \fntext[label2]{}
%% \cortext[cor1]{}
%% \address{Address\fnref{label3}}
%% \fntext[label3]{}

\title{Title/Name of your software\todo{This is the version with comments. To disable comments, comment out line~3 in the \LaTeX{} source.}}

%% use optional labels to link authors explicitly to addresses:
%% \author[label1,label2]{}
%% \address[label1]{}
%% \address[label2]{}

\author[labelBG]{Bineet Ghosh}
\address[labelBG]{Your institute, some address}

\author[labelEA]{Étienne André}
\address[labelEA]{Université de Lorraine, CNRS, Inria, LORIA, F-54000 Nancy, France}

\begin{abstract}
%% Text of abstract
ca. 100 words

\end{abstract}

\begin{keyword}
%% keywords here, in the form: keyword \sep keyword - maximum of six
keyword 1 \sep keyword 2 \sep keyword 3
\end{keyword}

\end{frontmatter}

\linenumbers

\ea{hi}
\bg{hi}

%%%%%%%%%%%%%%%%%%%%%%%%%%%%%%%%%%%%%%%%%%%%%%%%%%%%%%%%%%%%
%%%%%%%%%%%%%%%%%%%%%%%%%%%%%%%%%%%%%%%%%%%%%%%%%%%%%%%%%%%%
\section{Introduction}
%%%%%%%%%%%%%%%%%%%%%%%%%%%%%%%%%%%%%%%%%%%%%%%%%%%%%%%%%%%%
%%%%%%%%%%%%%%%%%%%%%%%%%%%%%%%%%%%%%%%%%%%%%%%%%%%%%%%%%%%%

\cite{GA22}

\instructions{
This ancillary data table is required for the sub-version of the codebase.
Please replace the text in the right column with the correct information
about your current code and leave the left column untouched.
}

\begin{table}[!h]
	\begin{tabular}{|l|p{6.5cm}|p{6.5cm}|}
	\hline
	\textbf{Nr.} & \textbf{Code metadata description} & \textbf{Please fill in this column} \\
	\hline
	C1 & Current code version & For example v42 \\
	\hline
	C2 & Permanent link to code/repository used for this code version & For example: $https://github.com/mozart/mozart2$ \\
	\hline
	C3  & Permanent link to Reproducible Capsule & \\
	\hline
	C4 & Legal Code License   & List one of the approved licenses \\
	\hline
	C5 & Code versioning system used & For example svn, git, mercurial, etc. put none if none \\
	\hline
	C6 & Software code languages, tools, and services used & For example C++, python, r, MPI, OpenCL, etc. \\
	\hline
	C7 & Compilation requirements, operating environments and dependencies & \\
	\hline
	C8 & If available, link to developer documentation/manual & For example: $http://mozart.github.io/documentation/$ \\
	\hline
	C9 & Support email for questions & \\
	\hline
	\end{tabular}
	\caption{Code metadata (mandatory)}
	\label{table:metadata}
\end{table}

%%%%%%%%%%%%%%%%%%%%%%%%%%%%%%%%%%%%%%%%%%%%%%%%%%%%%%%%%%%%
%%%%%%%%%%%%%%%%%%%%%%%%%%%%%%%%%%%%%%%%%%%%%%%%%%%%%%%%%%%%
\section{Motivation and significance}
%%%%%%%%%%%%%%%%%%%%%%%%%%%%%%%%%%%%%%%%%%%%%%%%%%%%%%%%%%%%
%%%%%%%%%%%%%%%%%%%%%%%%%%%%%%%%%%%%%%%%%%%%%%%%%%%%%%%%%%%%

\instructions{
In this section, we want you to introduce the scientific background and the
motivation for developing the software.

\begin{itemize}
  \item Explain why the software is important and describe the exact
      (scientific) problem(s) it solves.

  \item Indicate in what way the software has contributed (or will contribute
      in the future) to the process of scientific discovery; if available,
      please cite a research paper using the software.

  \item Provide a description of the experimental setting. (How does the user
      use the software?)

  \item Introduce related work in literature (cite or list algorithms used,
      other software, and so on).
\end{itemize}
}

%%%%%%%%%%%%%%%%%%%%%%%%%%%%%%%%%%%%%%%%%%%%%%%%%%%%%%%%%%%%
%%%%%%%%%%%%%%%%%%%%%%%%%%%%%%%%%%%%%%%%%%%%%%%%%%%%%%%%%%%%
\section{Software description}
%%%%%%%%%%%%%%%%%%%%%%%%%%%%%%%%%%%%%%%%%%%%%%%%%%%%%%%%%%%%
%%%%%%%%%%%%%%%%%%%%%%%%%%%%%%%%%%%%%%%%%%%%%%%%%%%%%%%%%%%%

\instructions{
Describe the software. Provide enough detail to help the reader understand
its impact.
}

%%%%%%%%%%%%%%%%%%%%%%%%%%%%%%%%%%%%%%%%%%%%%%%%%%%%%%%%%%%%
\subsection{Software architecture}
%%%%%%%%%%%%%%%%%%%%%%%%%%%%%%%%%%%%%%%%%%%%%%%%%%%%%%%%%%%%

\instructions{
Give a short overview of the overall software architecture; provide a
pictorial overview where possible; for example, an image showing the
components. If necessary, provide implementation details.
}

%%%%%%%%%%%%%%%%%%%%%%%%%%%%%%%%%%%%%%%%%%%%%%%%%%%%%%%%%%%%
\subsection{Software functionalities}
%%%%%%%%%%%%%%%%%%%%%%%%%%%%%%%%%%%%%%%%%%%%%%%%%%%%%%%%%%%%

\instructions{
Present the major functionalities of the software.
}
%%%%%%%%%%%%%%%%%%%%%%%%%%%%%%%%%%%%%%%%%%%%%%%%%%%%%%%%%%%%
\subsection{Sample code snippets analysis or use cases of the software (optional)}
%%%%%%%%%%%%%%%%%%%%%%%%%%%%%%%%%%%%%%%%%%%%%%%%%%%%%%%%%%%%

%%%%%%%%%%%%%%%%%%%%%%%%%%%%%%%%%%%%%%%%%%%%%%%%%%%%%%%%%%%%
%%%%%%%%%%%%%%%%%%%%%%%%%%%%%%%%%%%%%%%%%%%%%%%%%%%%%%%%%%%%
\section{Illustrative examples}
%%%%%%%%%%%%%%%%%%%%%%%%%%%%%%%%%%%%%%%%%%%%%%%%%%%%%%%%%%%%
%%%%%%%%%%%%%%%%%%%%%%%%%%%%%%%%%%%%%%%%%%%%%%%%%%%%%%%%%%%%

\instructions{
Provide at least one illustrative example to demonstrate the major functions
of your software/code. If you wish to include a video to supplement your
Original Software Publication, please ensure the file is included as
supplementary material or provide a link to your video in this
section.
This section can also refer to/summarize already published examples and case studies.
}
%%%%%%%%%%%%%%%%%%%%%%%%%%%%%%%%%%%%%%%%%%%%%%%%%%%%%%%%%%%%
%%%%%%%%%%%%%%%%%%%%%%%%%%%%%%%%%%%%%%%%%%%%%%%%%%%%%%%%%%%%
\section{Impact}
%%%%%%%%%%%%%%%%%%%%%%%%%%%%%%%%%%%%%%%%%%%%%%%%%%%%%%%%%%%%
%%%%%%%%%%%%%%%%%%%%%%%%%%%%%%%%%%%%%%%%%%%%%%%%%%%%%%%%%%%%

\instructions{
This is the main section of the article and reviewers will weigh it
appropriately. Please indicate:

\begin{itemize}
  \item Any new research questions that can be pursued as a result of your
      software.

  \item In what way, and to what extent, your software improves the pursuit
      of existing research questions.

  \item Any ways in which your software has changed the daily practice of
      its users.

  \item If applicable, how widespread the use of the software is within and
      outside the intended user group (downloads, number of users if your
      software is a service, citable publications, and so on).

  \item If applicable, how the software is being used in commercial
      settings or how it has led to the creation of spin-off companies.
      Please note that points 1 and 2 are best demonstrated by references
      to citable publications. \end{itemize}
}

%%%%%%%%%%%%%%%%%%%%%%%%%%%%%%%%%%%%%%%%%%%%%%%%%%%%%%%%%%%%
%%%%%%%%%%%%%%%%%%%%%%%%%%%%%%%%%%%%%%%%%%%%%%%%%%%%%%%%%%%%
\section{Conclusions}
%%%%%%%%%%%%%%%%%%%%%%%%%%%%%%%%%%%%%%%%%%%%%%%%%%%%%%%%%%%%
%%%%%%%%%%%%%%%%%%%%%%%%%%%%%%%%%%%%%%%%%%%%%%%%%%%%%%%%%%%%

%%%%%%%%%%%%%%%%%%%%%%%%%%%%%%%%%%%%%%%%%%%%%%%%%%%%%%%%%%%%
%%%%%%%%%%%%%%%%%%%%%%%%%%%%%%%%%%%%%%%%%%%%%%%%%%%%%%%%%%%%
\section{Future Plans (optional)}
%%%%%%%%%%%%%%%%%%%%%%%%%%%%%%%%%%%%%%%%%%%%%%%%%%%%%%%%%%%%
%%%%%%%%%%%%%%%%%%%%%%%%%%%%%%%%%%%%%%%%%%%%%%%%%%%%%%%%%%%%
\instructions{
Your future plans to expand the software.
}

%%%%%%%%%%%%%%%%%%%%%%%%%%%%%%%%%%%%%%%%%%%%%%%%%%%%%%%%%%%%
%%%%%%%%%%%%%%%%%%%%%%%%%%%%%%%%%%%%%%%%%%%%%%%%%%%%%%%%%%%%
\section*{Acknowledgements (optional)}
%%%%%%%%%%%%%%%%%%%%%%%%%%%%%%%%%%%%%%%%%%%%%%%%%%%%%%%%%%%%
%%%%%%%%%%%%%%%%%%%%%%%%%%%%%%%%%%%%%%%%%%%%%%%%%%%%%%%%%%%%

This work was partially supported by Bineet Ghosh's Chateaubriand fellowship.

\instructions{
You can use this section to acknowledge colleagues who do not qualify as a
co-authors but helped you in some way. Optionally thank people and institutes
you need to acknowledge.
}

\instructions{
If the software repository you used supplied a DOI or another Persistent
IDentifier (PID), please add a reference for your software here. For more
guidance on software citation, please see our guide for authors or this
article on the essentials of software citation by FORCE 11, of which Elsevier
is a member.
}
% \section*{References}

%% References:
%% If you have bibdatabase file and want bibtex to generate the
%% bibitems, please use
%%

\newcommand{\CCIS}{Communications in Computer and Information Science}
\newcommand{\ENTCS}{Electronic Notes in Theoretical Computer Science}
\newcommand{\FAC}{Formal Aspects of Computing}
\newcommand{\FundInf}{Fundamenta Informaticae}
\newcommand{\FMSD}{Formal Methods in System Design}
\newcommand{\IJFCS}{International Journal of Foundations of Computer Science}
\newcommand{\IJSSE}{International Journal of Secure Software Engineering}
\newcommand{\IPL}{Information Processing Letters}
\newcommand{\JAIR}{Journal of Artificial Intelligence Research}
\newcommand{\JLAP}{Journal of Logic and Algebraic Programming}
\newcommand{\JLAMP}{Journal of Logical and Algebraic Methods in Programming} % New name of JLAP since 2014
\newcommand{\JLC}{Journal of Logic and Computation}
\newcommand{\LMCS}{Logical Methods in Computer Science}
\newcommand{\LNCS}{Lecture Notes in Computer Science}
\newcommand{\RESS}{Reliability Engineering \& System Safety}
\newcommand{\STTT}{International Journal on Software Tools for Technology Transfer}
\newcommand{\TCS}{Theoretical Computer Science}
\newcommand{\ToPNoC}{Transactions on Petri Nets and Other Models of Concurrency}
\newcommand{\TSE}{{IEEE} Transactions on Software Engineering}

\bibliographystyle{elsarticle-num}
\bibliography{artifact}

%% else use the following coding to input the bibitems directly in the
%% TeX file.

%\begin{thebibliography}{00}
%
%%% \bibitem{label}
%%% Text of bibliographic item
%
%\bibitem{}
%
%\end{thebibliography}

\end{document}
\endinput
%%
%% End of file `SoftwareX_article_template.tex'.
